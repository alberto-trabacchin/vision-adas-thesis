\chapter{Introduction}
\pagenumbering{arabic}

\section{\acl{adas}}
Advanced Driver Assistance Systems (ADAS) are technologies that improve road 
safety by enhancing both vehicle and driver capabilities. These systems support 
the driver in the driving process, reducing the likelihood of human error and 
increasing road safety. ADAS includes a range of features from basic functions 
such as automatic headlights and rain-sensing windshield wipers to more complex 
systems like adaptive cruise control, lane departure warning, and collision 
avoidance systems.

The evolution of ADAS has been fueled by advancements in sensors, computing 
power, and connectivity. These technologies work together to provide vehicles 
with the ability to not only sense their environment but also to analyze and 
respond to potential hazards. As such, ADAS is seen as a crucial step towards 
fully autonomous vehicles, providing a safer, more comfortable driving experience.

\subsection{Market Size and Growth}
The market for \ac{adas} is experiencing robust growth as these technologies 
become more integral to new vehicle designs, emphasizing safety and driving 
efficiency. 

In 2024, the global ADAS market is projected to be worth around 
\$64.05 billion, with expectations to expand significantly at a \ac{cagr} 
of 12.7\% reaching approximately \$211.71 billion by 2034 \cite{adas_report_2023}.
This growth is given by advancements in autonomous driving technology and 
an increasing emphasis on vehicle safety from both consumers and functional 
safety certifications. 
Significant investments are also being made in the development and 
implementation of ADAS technologies across various global markets, including 
Japan, China, the United States, and Europe \cite{adas_report_2023}.

\begin{table}[h]
    \centering
    \begin{tabular}{r|c}
        \hline
        \textbf{Market} & \textbf{CAGR} \\
        \hline
        Worldwide & 12.7\% \\
        Japan & 13.6\% \\
        China & 13.5\% \\
        United States & 12.5\% \\
        Canada & 9.7\% \\
        Germany & 8.8\% \\
        Spain & 8.5\% \\
        \hline
    \end{tabular}
    \caption[CAGR of the ADAS market in different countries]
    {\ac{cagr} of the ADAS market in different countries \cite{adas_report_2023}.}
    \label{tab:adas_revenue}
\end{table}
Considering the market value and growth rate of \ac{adas} technologies, 
ADAS systems must also be economically viable on a large scale. 
This necessitates the use of cost-effective sensors such as cameras and radars, 
which offer a balance between performance and affordability. While there are 
more accurate options available, such as LiDAR sensors, these are typically more 
expensive and are primarily used for self-driving car experiments in research 
laboratories. The high cost of LiDAR sensors makes them less suitable for 
widespread deployment in consumer vehicles. Therefore, the challenge lies in 
leveraging affordable technologies like cameras and radars to their maximum 
potential, to deliver effective ADAS systems that can be widely adopted.

\subsection{Market Ecosystem}
The market ecosystem for \ac{adas} can be 
conceptually divided into several layers, each representing a different segment 
of the value chain. These layers range from \ac*{oems} to the suppliers of the key 
components such as sensors and processors. 
Here's a breakdown of these layers and the main companies in each field:

\subsubsection{\ac{oems}}
\acl{oems} are companies that integrate ADAS into their vehicles. They either develop 
some of their own ADAS technologies or incorporate systems designed by 
Tier 1 suppliers. Some of the leading OEMs in the ADAS market include 
Honda, Toyota, Nissan, General Motors, Tesla, Ford, Volkswagen, Volvo, BMW, and 
Mercedes-Benz.

\subsubsection{Tier 1 Suppliers}
Tier 1 suppliers develop complete systems or significant components that are 
directly supplied to \ac{oems}. These companies often develop the software and 
hardware integration necessary for ADAS.
Main Tier 1 suppliers in the ADAS market include Bosch, Continental, Aptiv, 
Denso, Magna, and Valeo.

\subsubsection{Autonomous Technology Developers}
This layer includes companies specifically focused on developing software and 
platforms that enable autonomous driving capabilities beyond standard ADAS features.
Key players in this segment include Waymo (Google/Alphabet), Cruise (General Motors),
Argo AI (Ford, Volkswagen), Aurora, and Zoox (Amazon).

\subsubsection{Sensing and Perception}
Companies in this segment focus on developing sensors and perception technologies 
that allow vehicles to perceive their environment, which is essential for both 
ADAS and fully autonomous functionalities.
Main providers are Luminar (LiDARs), Velodyne (LiDARs), Mobileye (cameras), 
Foresight Autonomous (cameras).

\subsubsection{Processors and Computing}
This segment involves the manufacturers of the advanced computing systems that 
process inputs from various sensors to make real-time driving decisions.
Key players include NVIDIA (GPUs and SoCs), Intel (through Mobileye), 
Qualcomm (processors), NXP Semiconductors (microcontrollers), and 
Renesas (microcontrollers and SoCs).

\subsubsection{Sensors}
This layer includes manufacturers of the various types of sensors used in ADAS 
and autonomous vehicles, such as cameras, radar, LiDAR, and ultrasonic sensors.
Main companies in this segment include Bosch (radar and video sensors),
Continental (radar sensors), Sony (image sensors), Texas Instruments (radar chips),
and ON Semiconductor (image sensors for automotive cameras).


\subsection{Key Components of \ac{adas}}
\ac{adas} systems are composed of several key components that work together to 
enhance vehicle safety and driving experience. These components include sensors, 
control units, actuators, human-machine interface, and connectivity.

\subsubsection{Sensors}
Sensors help the vehicle perceive its environment by collecting data on the 
surrounding objects and road conditions. The main types of sensors used in ADAS 
systems are:

\begin{itemize}
    \item \textbf{Radar Sensors}: These sensors use radio waves to detect the 
    distance and speed of objects around the vehicle. They are particularly 
    useful for measuring the relative speed of other vehicles and detecting 
    objects in poor weather conditions.
    
    \item \textbf{Cameras}: Cameras provide visual data that is used for object 
    recognition, lane detection, traffic sign recognition, and other visual 
    tasks. They are essential for understanding the vehicle's surroundings and 
    identifying potential hazards.
    
    \item \textbf{Ultrasonic Sensors}: These sensors are used for close-range 
    detection, primarily in parking assistance systems. They help the vehicle 
    detect obstacles when maneuvering at low speeds.
    
    \item \textbf{LiDAR}: LiDAR sensors use laser pulses to create a 3D map of 
    the vehicle's surroundings. They are particularly useful for creating 
    detailed maps of the environment and detecting objects at longer ranges.
\end{itemize}

\subsubsection{Control Units}
Control units are responsible for processing the data from sensors and cameras, 
making real-time decisions, and sending commands to the vehicle's actuators.
Decision making algorithms are implemented in these units to interpret sensor 
data and determine the appropriate response to different driving scenarios.

\subsubsection{Actuators}
Actuators are the components that control the vehicle's braking, steering, and 
acceleration systems based on the commands received from the control units. 
These actuators are responsible for executing the decisions made by the ADAS 
to ensure safe and efficient driving.

\subsubsection{Human-Machine Interface}
The human-machine interface is the system through which the driver interacts 
with the ADAS. This interface includes displays, alerts, and notifications that 
inform the driver about the status of the vehicle and provide warnings or 
assistance when needed. The interface is crucial for ensuring that the driver 
remains aware of the vehicle's behavior and can intervene when necessary.

\subsubsection{Connectivity}
Connectivity is an essential component of modern ADAS systems, enabling 
communication between the vehicle and external systems. This connectivity allows 
the vehicle to receive real-time traffic information, software updates, and 
remote assistance. It also enables vehicle-to-vehicle and vehicle-to-infrastructure 
communication, which is crucial for advanced safety features like collision 
avoidance and traffic management.


\subsection{Common \ac{adas} Features}
ADAS systems offer a wide range of features that enhance vehicle safety, 
comfort, and efficiency. Some of the most common ADAS features include: 
\ac*{acc}, \ac*{ldw}, \ac*{bsd}, \ac*{aeb}, \ac*{tsr}, \ac*{dms}, parking assistance,
adaptive headlights, and more. Most important functionalities are described in detail.

\subsubsection{\ac{acc}}
\ac{acc} is a feature that maintains a set speed and adjusts it to keep a safe 
distance from the vehicle ahead. It uses sensors to detect the distance and 
speed of the vehicle in front and automatically adjusts the vehicle's speed to
maintain a safe following distance.

\subsubsection{\ac{ldw}}
\ac{ldw} is a system that alerts the driver if the vehicle begins to drift out 
of its lane without signaling. It uses cameras or sensors to monitor the vehicle's 
position within the lane and provides visual or audible warnings if the vehicle 
starts to veer off course. Some vehicles also have \ac{lka} systems that can 
automatically steer the vehicle back into its lane if the driver does not respond 
to the warnings.

\subsubsection{\ac{bsd}}
\ac{bsd} is a system that warns the driver of vehicles in the blind spot during
lane changes. It uses sensors to detect vehicles in adjacent lanes that may not
be visible in the side mirrors and provides visual or audible alerts to prevent
collisions during lane changes.

\subsubsection{\ac{aeb}}
\ac{aeb} is a safety feature that detects imminent collisions with vehicles,
pedestrians, or other obstacles and automatically applies the brakes to prevent
or mitigate the impact. This system helps reduce the severity of accidents by
providing an additional layer of protection when the driver fails to react in
time.

\subsubsection{\ac{tsr}}
\ac{tsr} is a feature that uses cameras or sensors to identify traffic signs
such as speed limits, stop signs, and road markings. It displays this information
on the vehicle's dashboard or head-up display, helping the driver stay informed
about the current road conditions and regulations.

\subsubsection{\ac{dms}}
\ac{dms} is a system that monitors the driver's attention and alertness while
driving. It uses sensors to track the driver's eye movements, head position, and
other behavioral cues to detect signs of drowsiness, distraction, or inattention.
The system can provide warnings or take corrective actions to prevent accidents
caused by driver fatigue or distraction.
%
\section{Focus of this work}
%
In the field of modern automotive technology, \ac{adas} are playing a keyrole at augmenting driver safety, comfort, and overall driving experience.
These systems, leveraging advancements in computer vision, sensor technologies, and artificial intelligence, are reshaping the automotive industry introducing semi-autonomous and autonomous driving capabilities.

According to the \ac{who} road traffic injuries are among the leading causes of death worldwide, with an estimated 1.19 million fatalities annually.
With the aim of contributing to this transformative field, in this thesis various computer vision algorithms are implemented and evaluated, analyzing their strengths and weaknesses in addressing specific driving tasks.
From tracking of vulnerable users, such as pedestrians and cyclists, to the comprehension of dynamic driving environments, this research highlights advantages and limitations of each approach in real-world scenarios.
Further considerations are made on the trade-offs between complexity of the tasks and available data to train the models. 

By scrutinizing the performance of these algorithms in real-world scenarios and carefully assessing their respective advantages and limitations, this thesis aims to provide valuable insights into their applicability and efficacy within the domain of \ac{adas}. 
Through this rigorous evaluation process, a nuanced understanding of the suitability of different computer vision techniques for specific driving tasks is achieved, paving the way for informed decision-making and future advancements in automotive safety and technology.

In the pursuit of contributing to this transformative field, this thesis endeavors to explore and implement a computer vision-based approach for \ac{adas} systems. 
Specifically, the focus is on devising methodologies to enhance perception and decision-making capabilities within the realm of \ac{adas} through the lens of computer vision.

This thesis is divided into two main approaches, each addressing distinct facets of \ac{adas} functionality.
The first approach revolves around a traditional computer vision methodology, leveraging principles of multiple view geometry and object tracking. 
By harnessing the fundamentals of geometric and visual analysis, this approach aims to provide robust and reliable solutions for tasks such as lane detection, object recognition, and tracking within the \ac{adas} framework.

The second approach represents a departure from conventional techniques, delving into the realm of deep learning and vision transformers. 
Here, the emphasis shifts towards a more generalized approach to video-frame classification, leveraging the power of deep neural networks and semi-supervised learning techniques. 
By harnessing the capacity of vision transformers to capture intricate spatial and temporal features from raw video data, this approach seeks to push the boundaries of \ac{adas} capabilities, particularly in scenarios involving complex and dynamic environments.

Through the amalgamation of these two distinct approaches, this thesis endeavors to contribute to the ongoing evolution of \ac{adas} systems, striving towards the realization of safer, more efficient, and ultimately autonomous driving experiences. 
By exploring the synergy between traditional computer vision methodologies and cutting-edge deep learning techniques, this work aims to unlock new directions for innovation and advancement within the realm of automotive technology.

\section{Thesis Organization}
