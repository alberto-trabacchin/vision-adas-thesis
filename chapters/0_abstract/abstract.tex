\thispagestyle{empty}
\section*{Abstract}
\vspace{0.5cm}
This thesis presents a comprehensive study on the development of a computer 
vision-based system for Advanced Driver Assistance Systems (ADAS). The research 
initially explored a classical computer vision approach, which involved 
employing detection and tracking algorithms to perceive the external 
environment and utilizing gaze projection techniques to monitor the driver’s 
gaze. The primary objective was to enhance road safety by comparing the 
driver’s gaze with the locations of vulnerable road users, thereby proposing 
an initial safety scheme.

Despite the promising conceptual framework, it was observed that many of the 
conventional methods for extracting indirect features were not sufficiently 
robust in real-world driving scenarios. These methods struggled particularly 
under varying light conditions and during critical situations, highlighting 
significant limitations. In response to these challenges, the research 
transitioned to a deep learning-based approach. The core investigation centered 
on the capabilities of a Vision Transformer in extracting human decision-making 
biases inherent in driving tasks. This approach was further augmented by 
employing semi-supervised learning techniques, which enabled the effective 
utilization of vast amounts of easily accessible unlabeled data, thus 
addressing the challenges associated with the expensive and labor-intensive 
labeling process.

The outcomes of this research illustrate the substantial potential of deep 
learning models, particularly Vision Transformers, in enhancing the robustness 
and reliability of ADAS. The study also opens up new avenues for future 
research by identifying and addressing the challenges encountered during the 
development process. These findings underscore the critical role of computer 
vision in the advancement of ADAS, emphasizing its significance in improving 
driver assistance systems through more accurate and dependable perception 
mechanisms.

In conclusion, this thesis contributes to the field of ADAS by demonstrating 
how modern computer vision techniques can be effectively integrated into driver 
assistance systems. It highlights the transformative potential of deep 
learning, especially in overcoming the limitations of traditional methods, and 
paves the way for future innovations aimed at achieving safer and more 
efficient driving experiences.

\afterpage{\blankpage}
