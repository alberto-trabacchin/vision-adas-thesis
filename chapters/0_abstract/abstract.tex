\thispagestyle{empty}
\section*{Abstract}
\vspace{0.5cm}
This thesis presents a comprehensive study on the development of a computer 
vision-based system for \acf{adas}. The research 
initially explored a classical computer vision approach, which involved 
employing detection and tracking algorithms and monocular depth estimation
to perceive the external environment. Moreover, the study focused on 
integrating the driver’s attention state through its gaze projected from a pair 
of eye-tracking glasses onto the external scene, through a roof-mounted camera.
The primary objective was to analyze the driver's behavior by comparing 
its gaze with the locations of vulnerable road users, thereby proposing 
an initial safety scheme.

Despite the promising conceptual framework, it was observed that many of the 
conventional methods for extracting indirect features were not sufficiently 
robust in real-world driving scenarios. These methods struggled particularly 
under varying light conditions and during critical situations, highlighting 
significant limitations. In response to these challenges, the research 
transitioned to a deep learning-based approach. The core investigation centered 
on the capabilities of a \acf{vit} in extracting human decision-making 
biases inherent in detecting dangerous driving situations.
This approach was further improved by 
employing semi-supervised learning techniques, which leverage the 
vast amounts of easily accessible unlabeled data, thus 
addressing the challenges associated with the expensive and labor-intensive 
labeling process.

The outcomes of this research shows the substantial potential of 
the attention mechanism, on which vision transformer is based, in enhancing 
the robustness 
and reliability of \acs{adas}. The study also opens up new themes for future 
research by identifying and addressing the challenges encountered during the 
development process. These findings underscore the critical role of computer 
vision in the advancement of \acs{adas}, emphasizing its significance in improving 
driver assistance systems through more accurate and reliable perception 
mechanisms.

In conclusion, this thesis contributes to the field of ADAS by demonstrating 
how modern computer vision techniques can be effectively integrated into driver 
assistance systems. It highlights the potential of deep 
learning, especially in overcoming the limitations of traditional methods, and 
points to future innovations for safer and more 
efficient driving experiences.

\afterpage{\blankpage}
